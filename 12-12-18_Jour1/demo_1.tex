\documentclass[a4paper, twoside, 10pt]{article}
\usepackage[utf8]{inputenc} % Usage des caractères spéciaux
\usepackage{fontenc}[T1]
\usepackage{enumerate} % Options supplémentaires pour les énumérations.
\usepackage[francais]{babel} % respect des normes françaises (table des matières etc...)
\usepackage{amsmath, amsfonts} % Packages mathématiques

\author{Ludovic}
\title{Une première prise en main de \LaTeX}
\date{Un jour en décembre}
%\setlength{\baselineskip}{2cm}

\begin{document}
\maketitle % titre
\tableofcontents % table des matières

\section{Une première partie} % Un titre de partie

\subsection{Une sous partie}

\subsubsection{Les énumérations}

\begin{itemize}
\item Une chose,
\item Une autre chose,
\item Une dernière chose.
\end{itemize}

\begin{enumerate}[i)]
\item Une chose,
\item Une autre chose,
\item Une dernière chose.
\end{enumerate}

\subsection{Les paragraphes}

%\noindent 
Du texte très intéressant mais aussi super profond qui veut dire des choses sur la vie en général mais aussi sur des cas particuliers. 
Du texte très intéressant mais aussi super profond qui veut dire des choses sur la vie en général mais aussi sur des cas particuliers.
Du texte très intéressant mais aussi super profond qui veut dire des choses sur la vie en général mais aussi sur des cas particuliers.
On aime les mots longs anticonstitutionellement anticonstitutionellement anticonstitutionellement anticonstitutionellement.
Enfin bon voila.

%\vspace{2cm} % espace vertical

Un nouveau paragraphe. 

\section{Les mathématiques avec \LaTeX}

Il existe deux grands types d'environnements de maths. des maths en ligne. Exemple, la fonction $f(\alpha) \geq 5$ . 

$$
f(\alpha) = \underbrace{
			   \sum_{i = 0}^\infty \zeta_{ik}^5
			          }_{=0}		          
$$

\begin{equation}
\frac{3}{2} \times \frac{4}{3} = 2 \Xi \mathcal E \mathcal R \mathbb R
\label{eq:fractions}
\end{equation}

On peut associer des labels aux équations et à tous les objets. Par exemple, on peut citer l'équation \ref{eq:fractions} qui est à la page \pageref{eq:fractions}.

\subsection{Matrices et autres environnements}

\begin{equation}
M = 
\begin{pmatrix}
a & \frac{2}{3} \\
\dfrac{5}{2} & 3/4 
\end{pmatrix}
\end{equation}

$$
a = \left(
	\dfrac{\alpha + \dfrac{\beta}{\kappa}}{5+\upsilon}                                                               
	\right)
$$


$$
f(x) = \left\lbrace 
       \begin{split}
       5 , \forall x \in \mathbb R^* \\
       12 \forall x \in \lbrace 0 \rbrace  
       \end{split}
       \right.
$$

\begin{eqnarray}
a & = &  b + c \\
  & \neq & d + z
\end{eqnarray}

\section{Commandes}
\newcommand{\myvec}[1]{\vec{ \underline{\underline{#1}}} } % Une commande perso
\newcommand{\myvecc}[3]{ (#1, #2, #3) } % Une commande perso

Par exemple, on veut afficher des vecteurs comme $ \myvec v$ ou encore $\myvec{AB}$. Et aussi $\myvecc{1}{2}{3}$  et $\myvecc 456$

\end{document}